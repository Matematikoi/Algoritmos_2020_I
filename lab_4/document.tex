\documentclass[10pt,a4paper,onecolumn]{article}
\usepackage[utf8]{inputenc}
\usepackage[T1]{fontenc}
\usepackage[spanish]{babel}
\usepackage{amsmath}
\usepackage{amsfonts}
\usepackage{amssymb}
\usepackage{graphicx}
\usepackage[width=18.00cm, height=26.00cm]{geometry}
\title{Laboratorio 4\\Algoritmos}
\author{Gabriel Octavio Lozano Pinzón}

\begin{document}
\maketitle
\section{Obtener las mejores empresas para invertir}	
Deseamos escoger las empresas que nos darían la mayor ganancia si invirtieramos en ella en un tiempo anterior. Para este ejercicio manejamos $6$ meses y $2$ años. Esta tarea es una tarea repetitiva y monotona si se usan medios usuales como Yahoo Finance. Se puede simplificar bastantes usando un algoritmo en Quantopian que nos permitiría automatizar esta tarea para cualquier intervalo de tiempo y analizar el crecimiento de todas las empresas para las que hay información disponible al mismo tiempo.\\
Para resolver la tarea usamos un Notebook en los servicios provistos por Quantopian.  Comenzamos leyendo el universo de las empresas con el siguiente código.
\begin{figure}[h!]
	\centering
	\includegraphics[width=0.8\linewidth]{lectura_empresas}
	\caption {Código en Notebook de Quantopian para conocer la lista del universo de las empresas con las que se puede hacer trading.}
	\label{fig:lecturaempresas}
\end{figure}


\end{document}